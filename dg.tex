\documentclass[12pt]{article}
\usepackage{amsmath}
\usepackage{amssymb}
\usepackage{geometry}
\usepackage{fontspec}
\usepackage{enumerate}

% Page setup
\geometry{a4paper, margin=1in}

% --- Language Setup for Bengali ---
% NOTE: Compile with XeLaTeX.
% Ensure "Kalpurush" or your preferred Bengali font is installed.
\newfontfamily\bengalifont[Script=Bengali]{Kalpurush}
\newcommand{\textbn}[1]{{\bengalifont #1}}

% Paragraph spacing for better readability
\setlength{\parskip}{0.5em}

\title{\textbf{Differential Geometry} \\ }
\author{}
\date{}

\begin{document}

\maketitle

\begin{enumerate}

    % --- Part 1 ---

    \item \textbn{$\underline{x} = (a \cos u, a \sin u, bu)$ বক্ররেখার $u = \frac{\pi}{4}$ বিন্দুতে স্পর্শ লম্ব তলের সমীকরণ নির্ণয় কর।} \\
    \hspace*{\fill} \textbf{[Ch-1: Prob-13]}

    \item \textbn{প্রমাণ কর যে, $\underline{r} = \underline{r}(s)$ রেখাটি সরলরেখা হওয়ার প্রয়োজনীয় ও যথেষ্ট শর্ত হলো রেখাটির উপর সকল বিন্দুতে $k=0$।} \\
    \hspace*{\fill} \textbf{[Ch-1: Th-11]}

    \item \textbn{$\underline{x} = \underline{x}(s)$ বক্ররেখার ক্ষেত্রে Serret-Frenet ফর্মুলা বর্ণনাসহ প্রমাণ কর।} \\
    \hspace*{\fill} \textbf{[Ch-1: Art-1.24]}

    \item \textbn{প্রমাণ কর যে, $\underline{r} = \underline{r}(s)$ রেখা কুণ্ডলী হওয়ার প্রয়োজনীয় ও যথেষ্ট শর্ত $\frac{\tau}{k} = \pm \cot \alpha$, $\alpha$ ধ্রুবক।} \\
    \hspace*{\fill} \textbf{[Ch-1: Th-23]}

    \item \textbn{দেখাও যে, $\underline{r} = \underline{r}(s)$ বক্ররেখার ইভোলিউটের সমীকরণ:}
    \[
    \underline{r}_e = \underline{r} + \frac{\underline{n}}{k} + \frac{\underline{b}}{k} \cot \left(\int \tau \, ds + c\right)
    \]
    \textbn{যেখানে $c$ ইচ্ছামূলক ধ্রুবক। আরও দেখাও ইভোলিউটটি সমতলীয় বক্ররেখার কুণ্ডলী হবে।} \\
    \hspace*{\fill} \textbf{[Ch-1: Th-27]}

    \item \textbn{$x = a(t - \sin t), y = a(1 - \cos t), z = bt$ রেখার বক্রতা ও প্যাঁচ নির্ণয় কর।} \\
    \hspace*{\fill} \textbf{[Ch-1: Prob-41(a)]}

    \item \textbn{দেখাও যে, একটি বক্ররেখা $\underline{r} = \underline{r}(u)$ সমতলীয় বক্ররেখা হওয়ার প্রয়োজনীয় ও যথেষ্ট শর্ত হলো $[\dot{\underline{r}} \ \ddot{\underline{r}} \ \dddot{\underline{r}}] = 0$।} \\
    \hspace*{\fill} \textbf{[Ch-1: Th-19]}

    \item \textbn{দেখাও যে, বার্ট্রান্ড বক্ররেখাদ্বয়ের প্রতিষঙ্গী বিন্দুদ্বয়ের দূরত্ব ধ্রুবক।} \\
    \hspace*{\fill} \textbf{[Ch-1: Th-32]}

    \item \textbn{প্রমাণ কর যে,}
    \[
    \tau^2 = \frac{1}{\kappa^2}(\underline{x}''')^2 - \kappa^2 - \left(\frac{\kappa'}{\kappa}\right)^2
    \]
    \hspace*{\fill} \textbf{[Ch-1: Prob-46]}

    \item \textbn{প্রচলিত প্রতীকে প্রমাণ কর:}
    \[
    [\underline{b}' \ \underline{b}'' \ \underline{b}'''] = \tau^3 (\kappa' \tau - \kappa \tau')
    \]
    \hspace*{\fill} \textbf{[Ch-1: Prob-26]}

    \item \textbn{$f(u)$ নির্ণয় কর যেখানে $\underline{x} = (a \cos u, a \sin u, f(u))$ সমতলীয় বক্ররেখা।} \\
    \hspace*{\fill} \textbf{[Ch-1: Prob-55]}

    \item \textbn{দেখাও যে, $\underline{r} = (3u, 3u^2, 2u^3)$ বক্ররেখাটির যেকোনো বিন্দুতে স্পর্শকরেখা $y = z - x = 0$ রেখার সাথে ধ্রুবক কোণ উৎপন্ন করে।} \\
    \hspace*{\fill} \textbf{[Ch-1: Prob-11(i)]}

    \item \textbn{দেখাও যে, শুধুমাত্র ধ্রুবক অশূন্য বক্রতা বিশিষ্ট সমতলীয় রেখাই বৃত্ত হবে।} \\
    \hspace*{\fill} \textbf{[Ch-1: Prob-23]}

    \item \textbn{$\underline{x} = \underline{x}(s)$ বক্ররেখার $s$ বিন্দুতে দেখাও যে,}
    \[
    [\underline{x}'' \ \underline{x}''' \ \underline{x}^{iv}] = \kappa^5 \frac{d}{ds} \left(\frac{\tau}{\kappa}\right)
    \]
    \hspace*{\fill} \textbf{[Ch-1: Prob-25]}

    \item \textbn{প্রমাণ কর যে, একটি বক্ররেখা সরলরেখা হওয়ার প্রয়োজনীয় ও যথেষ্ট শর্ত হলো রেখাটির উপর সকল বিন্দুতে $k=0$।} \\
    \hspace*{\fill} \textbf{[Ch-1: Th-11]}

    \item \textbn{দেখাও যে, পর পর বিন্দুতে প্রধান লম্বসমূহ ছেদ করবে না যদি না $\tau=0$ হয়।} \\
    \hspace*{\fill} \textbf{[Ch-1: Prob-20]}

    \item \textbn{$x = 3t, y = 3t^2, z = 2t^3$ বক্ররেখার জন্য দেখাও যে,}
    \[
    \rho = \sigma = \frac{3}{2}(1 + 2t^2)^2
    \]

    \item \textbn{$\underline{r} = (a \cos 2u, a \sin 2u, 2a \sin u)$ বক্ররেখার বক্রতা ও প্যাঁচ নির্ণয় কর।}

    \item \textbn{যদি $\underline{x} = \underline{x}(s)$ বক্ররেখা, $a$ ব্যাসার্ধ এবং $c$ কেন্দ্রবিশিষ্ট কোনো গোলকের উপর অবস্থিত হয়, যার সমীকরণ $(\underline{y} - \underline{c}) \cdot (\underline{y} - \underline{c}) = a^2$, দেখাও যে,}
    \begin{enumerate}
        \item[(i)] $\underline{c} = \underline{x} + \frac{1}{k} \underline{n} - \frac{k'}{k^2 \tau} \underline{b}$ \\
        \hspace*{\fill} \textbf{[Ch-1: Prob-46]}
        
        \item[(ii)] $\frac{d}{ds} \left(\frac{k'}{k^2 \tau}\right) - \frac{\tau}{k} = 0$ \\
        \hspace*{\fill} \textbf{[Ch-1: Prob-47]}
        
        \item[(iii)] $\left(\frac{1}{k}\right)^2 + \left(\frac{k'}{k^2 \tau}\right)^2 = a^2$ \\
        \hspace*{\fill} \textbf{[Ch-1: Prob-48]}
    \end{enumerate}

    \setcounter{enumi}{22}

    \item \textbn{(খ) দেখাও যে, ইনভোলিউটের টরসান} $\tau_1 = \frac{k\tau' - k'\tau}{k(k^2 + \tau^2)(c - s)}$ । \\
    \hspace*{\fill} \textbf{[Ch-1: Th-26]}

    \item \textbn{$x^2y + 2xy = 4$ বক্ররেখার $(2, -2, 3)$ বিন্দুতে একক অভিলম্ব ভেক্টর নির্ণয় কর।} \\
    \hspace*{\fill} \textbf{[Ch-2: Prob-5(i) analogue]}

    \item \textbn{(i) $\underline{r} = \underline{r}(u, v)$ বক্রতলে প্রথম মৌলিক আকৃতি $I = ds^2 = E du^2 + 2F du dv + G dv^2$ হলে $E, F$ এবং $G$ নির্ণয় কর।} \\
    \hspace*{\fill} \textbf{[Ch-2: Art-2.14]}

    \item \textbn{(ii) প্রমাণ কর যে,} $EG - F^2 > 0$ \\
    \hspace*{\fill} \textbf{[Ch-2: Prob-6]}

    \item \textbn{দেখাও যে, $\underline{r} = \underline{r}(u, v)$ বক্রতলের ক্ষেত্রফল:}
    \[
    A = \iint_S \sqrt{EG - F^2} \, du \, dv
    \]
    \hspace*{\fill} \textbf{[Ch-2: Th-5]}

    \item \textbn{দেখাও যে, $x = r \cos \phi, y = r \sin \phi, z = a\phi + b$ বক্রতলের প্রথম মৌলিক আকারকে নিম্নোক্ত আকারে লেখা যায়:}
    \[
    ds^2 = dr^2 + (r^2 + a^2)d\phi^2
    \]
    \hspace*{\fill} \textbf{[Ch-2: Prob-13]}

    \item \textbn{$\underline{r} = (b \cos u + a \cos u \sin v, b \sin u + a \sin u \sin v, a \cos v)$ বক্রতলের ক্ষেত্রফল নির্ণয় কর। যেখানে $0 \le u \le 2\pi$ এবং $0 \le v \le 2\pi$} \\
    \hspace*{\fill} \textbf{[Ch-2: Prob-9]}

    \item \textbn{$z = f(x, y)$ বক্রতলের দ্বিতীয় মৌলিক আকার নির্ণয় কর।} \\
    \hspace*{\fill} \textbf{[Ch-2: Art-2.20]}

    % --- Part 2 ---

    \item \textbn{$\frac{x^2}{a^2} + \frac{y^2}{b^2} + \frac{z^2}{c^2} = 1$ উপবৃত্তকের আমবিক নির্ণয় কর।} \\
    \hspace*{\fill} \textbf{[Ch-3: Prob-9]}

    \item \textbn{(ক) উইনগার্টেন সমীকরণ ব্যবহার করে দেখাও যে,}
    \[
    [\underline{N} \ \underline{N}_u \ \underline{N}_v] = \frac{eg - f^2}{\sqrt{EG - F^2}}
    \]
    \hspace*{\fill} \textbf{[Ch-3: Prob-21]}

    \item \textbn{প্রধান বক্রতাসমূহ দেয় এমন সমীকরণ $(EG-F^2)\kappa_n^2 - (-2fF+gE+eG)\kappa_n + (eg-f^2) = 0$ প্রতিষ্ঠা কর।} \\
    \hspace*{\fill} \textbf{[Ch-3: Art-3.10]}

    \item \textbn{দেখাও যে, $\underline{r} = \underline{r}(u, v)$ তলের উপর প্রত্যেক বিন্দুতে একটি রেখা বক্রতার রেখা হওয়ার প্রয়োজনীয় শর্ত: $d\underline{N} + \kappa_n d\underline{r} = \underline{0}$ যেখানে $\kappa_n$ অভিলম্ব বক্রতা নির্দেশ করে।} \\
    \hspace*{\fill} \textbf{[Ch-3: Th-5]}

    \item \textbn{গাউসের সমীকরণটি প্রতিষ্ঠা কর।} \\
    \hspace*{\fill} \textbf{[Ch-3: Art-3.26]}

    \item \textbn{উইন গারটেনের সমীকরণটি প্রতিষ্ঠা কর।} \\
    \hspace*{\fill} \textbf{[Ch-3: Art-3.25]}

    \item \textbn{প্রমাণ কর যে, $III - 2MII + KI = 0$ , যেখানে}
    \begin{itemize}
        \item $III =$ \textbn{তৃতীয় মৌলিক আকার।}
        \item $II =$ \textbn{দ্বিতীয় মৌলিক আকার।}
        \item $I =$ \textbn{প্রথম মৌলিক আকার।}
        \item $M =$ \textbn{গড় বক্রতা।}
        \item $K =$ \textbn{গাউসীয় বক্রতা।}
    \end{itemize}
    \hspace*{\fill} \textbf{[Ch-3: Prob-15]}

\end{enumerate}

\end{document}
